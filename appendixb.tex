



We designed a network of 12 layers using \mintinline{c}/tensorflow.keras/in python, for the \diabased\ case as follows:
\vspace{1cm}

%\begin{figure*}[H]
\begin{minted}{python}
layer1 = keras.layers.Conv2D(16, kernel_size=(5, 5), padding="valid", activation="relu", 
                             input_shape=(51,153,1))
layer2 = keras.layers.MaxPooling2D((2, 2), strides=2)
layer3 = keras.layers.Dropout(0.4)
layer4 = keras.layers.Conv2D(32, kernel_size=(5, 5), padding="valid", activation="relu")
layer5 = keras.layers.MaxPooling2D((2, 2), strides=2)
layer6 = keras.layers.Dropout(0.4)
layer7 = keras.layers.Conv2D(64, kernel_size=(5, 5), padding="valid", activation="relu")
layer8 = keras.layers.MaxPooling2D((2, 2), strides=2)
layer9 = keras.layers.Dropout(0.4)
layer10 = keras.layers.Flatten()
layer11 = keras.layers.Dense(32, activation="relu")
layer12 = keras.layers.Dense(2, activation="softmax")

opt = keras.optimizers.SGD(learning_rate=0.01)
model.compile(optimizer=opt, loss="sparse_categorical_crossentropy", metrics=["accuracy"])

history = model.fit(feat_tr2, targ_tr, validation_split=0.20, epochs=50, batch_size=20)
\end{minted}
%\end{figure*}
% \FloatBarrier

\vspace{1cm}

\noindent and a network of 11 layers using  \mintinline{c}/tensorflow.keras/in python, for the \nodia\ case as follows:
\vspace{1cm}

\begin{minted}{python}
layer1 = keras.layers.Conv2D(1, kernel_size=(7, 7), padding="valid", activation="relu", 
                             input_shape=(51,102,1))
layer2 = keras.layers.MaxPooling2D((2, 2), strides=2)
layer3 = keras.layers.Conv2D(16, kernel_size=(3, 3), padding="valid", activation="relu")
layer4 = keras.layers.MaxPooling2D((2, 2), strides=2)
layer5 = keras.layers.Dropout(0.4)
layer6 = keras.layers.Conv2D(32, kernel_size=(3, 3), padding="valid", activation="relu")
layer7 = keras.layers.MaxPooling2D((2, 2), strides=2)
layer8 = keras.layers.Dropout(0.4)
layer9 = keras.layers.Flatten()
layer10 = keras.layers.Dense(32, activation="relu")
layer11 = keras.layers.Dense(2, activation="softmax")
\end{minted}
To compile the models we used the optimizer \mintinline{c}/SGD/ (stochastic gradient descent) with a \mintinline{c}/learning_rate=0.01/ the \mintinline{c}/"sparse_categorical_crossentropy"/ loss function and the \mintinline{c}/"accuracy"/ metric. Our models are available in a dedicated GitHub repository \footnote{\url{https://github.com/taceroc/DIA_noDIA}.}
