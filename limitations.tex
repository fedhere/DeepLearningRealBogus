This work is targeted to the investigation of CNN RB model performance, with and without \diff\ in input, on a single data set, the same dataset upon which the  development of the random-forest-based \texttt{autoscan} was based (see \autoref{sec:data}). This approach enables a straightforward comparison, but it comports some limitations. 

The labels in our data set come from simulations of SNe ($\texttt{label = 0}$) and visual inspection that classifies artifacts and moving objects ($\texttt{label = 1}$). We reserve to future work the investigation of the efficacy of the model on transients of different nature, including quasars (QSOs), strong lensed systems, Tidal Disruption Events (TDEs), and supernovae of different types.  These transients may have characteristically different associations with the host galaxy, including preferences for different galaxy types and locations with respect to the galaxy center, compared to the simulated SNe events in our training set. 

Specifically thinking of Rubin LSST data, an additional source of variability may be introduced by Differential Chromatic Refraction effects \citep{Abbott_2018, richards2018leveraging}, or stars with significant proper motion which, due to exquisite image quality of the Rubin images, would be detectable effects.

While we demonstrated CNN model's potential in the detection of transients without DIA, we did not address the question of completeness as a function of \search\ or \temp\ depth or the potential for performing accurate photometry without DIA. 

Improvements in the \nodia\ model performance could be achieved by considering %rotation invariance, 
alternative network architectures, more training epochs, tuned hyperparameters, {\it etc}., and we leave these tasks as future work beyond the proof-of-concept presented here. %The optimization of the architecture would be included for a future work. 

Finally, we note that our models did in fact implicitly leverage some of the information generated by the DIA even when they did not use the \diff\ image itself as input.  First, the \temp\ and \search\ images are de-warped. Since the transient alerts are generated from aligned DIA images, the transient source is always located at the image center in our data. We use postage stamps that were, however, not PSF matched or scaled to match the template brightness.
To move beyond a proof-of-concept, in future work we will re-train and apply our models to images whose alignment does not depend on the existence of a transient. %This would support strongly that the used of the \diff\ image is not necessary because nowadays the position of the source in the image-triplet is done based in the \diff\ image.