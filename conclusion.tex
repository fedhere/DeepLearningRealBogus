

In this work, we have demonstrated that high-accuracy models for classifying true astrophysical transients from artifacts and moving objects (a task generally known as ``real-bogus'') can be built without leveraging the results of a difference image analysis (DIA) pipeline that constructs a ``template-subtracted'' image.

Starting from the Dark Energy Survey dataset that supported the creation of the well known real-bogus \texttt{autoscan} model \citep{Goldstein_2015}, we first built a CNN-based model, dubbed \diabased, that uses a sky template (\temp), a nightly image (\search), and a template-subtracted version of the nightly image (\diff) that performs real-bogus classification at the level of \texttt{autoscan} model.
%state-of-the art models.  
Our \diabased\ model reaches a $97\%$ accuracy in the bogus classification with an Area-Under-the-Curve of 0.992 and does not require human decision in the feature engineering or extraction phase.

We then created \nodia, a model which uses only the \temp\ and \search\ images and can extract information that enables the identification of bogus transients with $92\%$ accuracy even with a naive model architecture that, to enable a direct comparison, is purposefully designed to be minimally different from our high accuracy \diabased\ model.

We further investigated what information enables the real-bogus classification in both the \diabased\ and \nodia\ models and demonstrated that a CNN trained with the DIA output primarily uses the information in the \diff\ image to make the final classification, and that the model examines a \diff-\search-\temp\ image-set fundamentally differently in the cases where there is a transient, than in the cases where there is not one. The \nodia\ model, conversely, takes a more comprehensive look at both \temp\ and \search\ images, but relies primarily on \temp\ to enable the reconstruction of the information found in the \diff.

Implementation of this methodology in future surveys could reduce the time and computational cost required for classifying transients by entirely omitting the construction of the difference images.  