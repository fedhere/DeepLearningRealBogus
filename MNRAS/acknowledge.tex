 
 
TAC and FBB acknowledge the support of the LSST Corporation and the Enabling Science Grant program that partially supported TAC through Grant No. 2021-040, Universidad Nacional de Colombia and University of Delaware for the Beyond Research Program (2020) and Summer Research Exchange (2021). MS and HQ were supported by DOE grant DE-FOA-0002424 and NSF grant AST-2108094. 

TAC: Prepared the data, conducted the analysis, created and maintains the machine learning models, wrote the manuscript.
FBB: Advised on data selection,  curation, and preparation,  selection and design, revised the manuscript.
GD: Advised on data preparation and model selection and design, revised the manuscript.
MS: Advised on data selection, shared knowledge about the compilation and curation of the original dataset, revised the manuscript.
HQ: Advised on data preparation and model design, revised the manuscript.


 This paper has undergone internal review in the LSST Dark Energy Science Collaboration.
 The internal reviewers were: Viviana Acquaviva, Suhail Dhawan, and Michael Wood-Vasey.
 
 
 
% This work used some telescope which is operated/funded by some agency or consortium or foundation ...
% We acknowledge the use of An-External-Tool-like-NED-or-ADS.
The DESC acknowledges ongoing support from the Institut National de 
Physique Nucl\'eaire et de Physique des Particules in France; the 
Science \& Technology Facilities Council in the United Kingdom; and the
Department of Energy, the National Science Foundation, and the LSST 
Corporation in the United States.  DESC uses resources of the IN2P3 
Computing Center (CC-IN2P3--Lyon/Villeurbanne - France) funded by the 
Centre National de la Recherche Scientifique; the National Energy 
Research Scientific Computing Center, a DOE Office of Science User 
Facility supported by the Office of Science of the U.S.\ Department of
Energy under Contract No.\ DE-AC02-05CH11231; STFC DiRAC HPC Facilities, 
funded by UK BIS National E-infrastructure capital grants; and the UK 
particle physics grid, supported by the GridPP Collaboration.  This 
work was performed in part under DOE Contract DE-AC02-76SF00515.

 This project used public archival data from the Dark Energy Survey (DES). Funding for the DES Projects has been provided by the U.S. Department of Energy, the U.S. National Science Foundation, the Ministry of Science and Education of Spain, the Science and Technology Facilities Council of the United Kingdom, the Higher Education Funding Council for England, the National Center for Supercomputing Applications at the University of Illinois at Urbana–Champaign, the Kavli Institute of Cosmological Physics at the University of Chicago, the Center for Cosmology and Astro-Particle Physics at the Ohio State University, the Mitchell Institute for Fundamental Physics and Astronomy at Texas A\&M University, Financiadora de Estudos e Projetos, Fundação Carlos Chagas Filho de Amparo à Pesquisa do Estado do Rio de Janeiro, Conselho Nacional de Desenvolvimento Científico e Tecnológico and the Ministério da Ciência, Tecnologia e Inovação, the Deutsche Forschungsgemeinschaft and the Collaborating Institutions in the Dark Energy Survey. The Collaborating Institutions are Argonne National Laboratory, the University of California at Santa Cruz, the University of Cambridge, Centro de Investigaciones Enérgeticas, Medioambientales y Tecnológicas–Madrid, the University of Chicago, University College London, the DES-Brazil Consortium, the University of Edinburgh, the Eidgenössische Technische Hochschule (ETH) Zürich, Fermi National Accelerator Laboratory, the University of Illinois at Urbana-Champaign, the Institut de Ciències de l'Espai (IEEC/CSIC), the Institut de Física d'Altes Energies, Lawrence Berkeley National Laboratory, the Ludwig-Maximilians Universität München and the associated Excellence Cluster Universe, the University of Michigan, the National Optical Astronomy Observatory, the University of Nottingham, The Ohio State University, the OzDES Membership Consortium, the University of Pennsylvania, the University of Portsmouth, SLAC National Accelerator Laboratory, Stanford University, the University of Sussex, and Texas A\&M University.

Based in part on observations at Cerro Tololo Inter-American Observatory, National Optical Astronomy Observatory, which is operated by the Association of Universities for Research in Astronomy (AURA) under a cooperative agreement with the National Science Foundation. 

This research used resources of the National Energy Research Scientific Computing Center (NERSC), a U.S. Department of Energy Office of Science User Facility located at Lawrence Berkeley National Laboratory, operated under Contract No. DE-AC02-05CH1123.

All of our code is written in \texttt{Python} using the following packages:
\begin{itemize}
\item \texttt{TensorFlow \citep{tensorflow2015-whitepaper}}
\item \texttt{Keras \citep{chollet2015keras}}
\item \texttt{numpy \citep{harris2020array}}
\item \texttt{astropy \citep{astropy:2013, astropy:2018} }
\item \texttt{pandas \citep{mckinney-proc-scipy-2010, reback2020pandas}}
\item \texttt{matplotlib \citep{Hunter:2007}}
\item \texttt{seaborn \citep{Waskom2021}}
\end{itemize}


\section{Data availability}
The data underlying this article are available in 
\url{https://portal.nersc.gov/project/dessn/autoscan/#tdata},
and explain in more detail in \cite{Goldstein_2015}.

